\chapter{Data Visualization}

Data Visualization is the visual representation of data using plots and visual models. In the infrastructural stack, it stands at the highest level since it's the mean through which the results of processing are shown to the final user. The reason that makes it so important is that humans are not able to extract useful information from raw data at a glance, and there comes the need for clear and intuitive plots which capture all the helpful statistics contained, in a concise way, from which humans gain insights and act accordingly.

The stack uses as a visualization tool \textbf{Tableau}, and in this chapter will be presented its structure and its strengths. 

\section{Introduction}

Tableau is a software dedicated to the creation of visualisations for the Business Intelligence. Tableau offers an highly intuitive and interactive way to the access, manipulation and analysis of data, without any need for coding, positioning itself among the first three product leaders in the sector \cite{gartner_tableau}. Tableau's main features include easiness and intuitiveness in the software use, data sources management and some innovative features such as table calculations and the possibility to use data analysis techniques directly on the plots.

Tableau On Premise version is composed of two elements:

\begin{itemize}
    \item \textbf{Tableau Desktop}: Allows for the creation of worksheets, dashboards and stories, linked to one or more data sources.
    \item \textbf{Tableau Server}: It is used to publish the plots created with Tableau Desktop, in order to make them available to the end users.
\end{itemize}

\section{Tableau Desktop}

Tableau is structured in a Microsoft Excel fashion, with workbooks and sheets: a \textbf{workbook} is made of many sheets which can be worksheets, dashboards or stories. A worksheet is a single plot, with its own filters and legend, and can be combined with other worksheets to make a dashboard. A story is a sequence of dashboards and it's usually the view shown to the end user. Each workbook starts from one or more data sources, used by Tableau for fetching the data used in the plots.

\subsection{Data Source}

The first step during the creation of a workbook is to link Tableau to a data source \cite{LearningTableau}, either through a \textbf{File Connection}, which supports all of the main data formats available such as JSON, CSV, TSV, PDF and Excel sheets, or a \textbf{Server Connection}, to \textbf{relational databases}, such as Microsoft SQL Server and Hive, \textbf{NoSQL databases}, such as MongoDB or Cassandra, and \textbf{data sources in cloud}, such as Google Analytics, Amazon Redshift and Salesforce. If there's no optimized built-in driver, it's possible to use any ODBC\footnote{ODBC: Open DataBase Connectivity is a standard API to access DBMS} driver. Any interaction with a data source is handled by default with \textbf{VizQL} (Visual Query Language), which translates all of the drag-n-drop actions executed in the GUI to adequate queries to the data source, in order to get the desired results available for visualisation.

Tableau's connection to a data source is available in two different types: Live and Extract.

\paragraph{Live}

A \textbf{Live} connection queries directly the data source, making the performances of this connection strictly dependent from the performances of the database being used. With this kind of connection it's possible to refresh all the plots each time data is being updated, making the visualisations almost real time.

\paragraph{Extract}

An \textbf{Extract} connection is a periodic query, which can be performed, for example, every day, which saves the results in a Tableau Data Extract file, which will be used for all of the following queries. The Extract may be considered a snapshot of the database and, for this very reason, any update in the database won't be seen until the next scheduled Extract File update. Even though the creation of an Extract File can take a long time, queries on it are very efficient because of its internal tabular structure, optimized for Tableau's own execution engine.

\subsection{View Creation}

Each field loaded from the data source is assigned one of two roles: dimension or measure. A \textbf{dimension} is used for categorised or discrete information, such as strings or dates, while a \textbf{measure} corresponds to numeric information, potentially object of aggregations. It's possible to change freely the roles of each field, according to the needs. Drag-n-drop allows to move dimensions and measures on both plots' axes and Tableau is able to automatically interpret the new layout, constructing a more significant chart.

\subsubsection{View Choice}
The main focus of visual analytics is to represent data in a fashion so that it conveys as much information as possible at a glance.
Tableau offers the possibility to create several kinds of visualisations, each one suitable for a specific use case. The three most common charts are without a doubt the bar graph for data comparisons, line graph for trends and pie charts to underline the share of different categories inside a single field.
More advanced representations are also supported such as:

\begin{itemize}
	\item Scatter plot, used to investigate correlation between variables.
	\item Gannt chart, used to describe scheduling in time.
	\item Bullet graph, used to give focus to our goal and its gap with the current state of affairs.
	\item Box and Whisker plot, used to highlight data distribution.
	\item Bubble chart, used to show the concentration of data.
	\item Tree map, used to emphasise hierarchy in a set.
	\item Geographical map, used to represent geo-coded data.
\end{itemize} 

\paragraph{Dual Axis}
While a graph may present more than one variable on a single axis, it is possible to place a secondary axis in a view to compare two variables along different scales. This technique is called \textbf{dual axis} and grants the ability to superimpose two independent visualisations upon the the same view.\\
Furthermore, it is possible to use dual axis with different kinds of charts to realize heterogeneous visualisations.

\subsection{Visual Analytics}

The feature that distinguishes a Visual Analytics solution from a plain visualisation application is the possibility to create interactive views and to process complex analyses.\\
Visual Analytics, in fact, describes a way to present information useful to spur visual thinking: data can be highlighted, filtered and quickly represented through new charts. This gives us the chance to execute new analyses inspired by our natural inclination towards thinking visually.

\subsubsection{Filtering}

Filtering is one of our main tools in data analysis and can be implemented in various layers of the Tableau stack.
A first filtering can be applied at the data source level: it is indeed possible to define a custom SQL query so that data from our source don't contain unneeded fields when imported into Tableau.\\
Once loaded on Tableau, contextual filtering can polish data before chart processing, an example of this would be a limitation on the number of records used for visualisation.\\
Lastly, filtering can be applied on measures and dimensions, once the view has been constructed. This filtering can either be statically applied by the developer or dynamically offered to the end user, thanks to special user interfaces, to make the view interactive.
\\
In more detail, once the dimension or measure to filter is chosen, it is necessary to pick a type of filtering to apply, such as single and multiple choice for dimensions, or exact value and range for measures. There is also a special section for dates which offers greater control over granularity.
Finally it is possible to filter data directly from the view selecting the markers of our interest.

\subsubsection{Table calculations}

\textbf{Table Calculations} are one of the main innovative features in Tableau: while all of the other operations are executed at the data source level, a table calculation is executed on the aggregated data composing the view, using Tableau's cache like the Figure \ref{fig:TableCalculation} shows.


\begin{figure}[ht]
    \begin{center}
        \includegraphics[width=0.8\linewidth]{Figures/TableCalculation}
    \end{center}
    \caption{Execution flow of a table calculation.}
    \label{fig:TableCalculation}
\end{figure}

Tableau gives the user the possibility to use a great variety of operations and allows for writing a custom table calculation, such as:

\begin{itemize}
    \item Computing the running total.
    \item Computing time variance in both absolute values and percentage.
    \item Computing the moving and percentile average.
    \item Ranking or sorting according to specific functions.
\end{itemize}

\subsubsection{Analysis}

Another innovative feature available in Tableau is the possibility to make statistical analyses through built-in mathematical models  or algorithms written in \textbf{R, Python or MatLab} \cite{LearningTableau}. The main built-in models usable in Tableau are:

\begin{itemize}
    \item \textbf{Trend lines}: each trend line added to a chart represents a regression, which Tableau builds according to linear, logarithmic, exponential or polynomial models.
    \item \textbf{Forecasting}: in case of the visualisation of a time series, it's possible to use the \textit{exponential smoothing} method to compute future values according to an exponential function, weighing current value and past values' average; Tableau can also recognize periodic patterns in time series.
    \item \textbf{Clustering}: Tableau uses K-Means algorithm to classify the data on the chosen variables and gives the user the possibility to change the value of K, that is the number of total clusters.
\end{itemize}

\section{Tableau Server}

\textbf{Tableau Server} is a server application used to contain data and metadata of sources and shared workbooks. Through an Access Control List based on accounts, Tableau server allows users to view, interact, download, share and edit a worksheet without the need to install Tableau Desktop. Additionally, through the sharing of an already present data source, a user can create new worksheets without installing the correspondent drivers.

\subsection{Tableau Embedded}

Tableau Server offers the possibility to embed worksheets in a web application using Tableau JavaScript API, which allows for:

\begin{itemize}
    \item Viewing and interacting with the shared view.
    \item Loading and resizing the view dynamically.
    \item Download the view as a PDF or an image.
\end{itemize}

Corresponding user credentials would be required to view the shared charts and plots, but it's possible to use Tableau's \textit{Trusted Authentication} to authenticate the server hosting the application in order to allow all the web app users access to it.