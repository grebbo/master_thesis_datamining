\chapter{Introduzione}

\section{I Big Data}
Con questo termine, sempre più diffuso nell'immaginario comune, si vuole indicare una grande mole di dati, strutturati e non, con il potenziale per ricavare ulteriori informazioni e indurre nuove riflessioni riguardo all'ambito considerato.\\
Sebbene la parola in sè sia stata coniata recentemente, l'ambiente che rappresentano ha origini molto più radicate e, con il tempo, si è diffusa la definizione descritta dalle tre \textbf{V}:

\begin{itemize}
	\item \textbf{Volume}: la quantità di dati manipolati deve essere enorme, non gestibile da un sistema tradizionale. Si parla di grandezze dell'ordine degli \textit{zettabytes}, ovvero $10^{21}$ bytes, che devono essere assunte dal sistema e memorizzate per l'analisi. Ogni dato può contenere risvolti rilevanti ed è fondamentale essere sicuri di non perdere nulla. 
	
	\item \textbf{Velocity}: i sistemi Big Data devono tenere il passo con i servizi che vengono loro richiesti. Applicazioni legate al mondo dell'IoT (Internet of Things) e alla sensorizzazione di alcune azioni quotidiane necessitano tempi di risposta veloci e ritmi di lavoro compassati.
	
	\item \textbf{Variety}: bisogna essere pronti a operare su dati eterogenei, in quanto la grande disponibilità di dati non implica un corrispondente aumento di dati strutturati. Dunque, oltre ai classici databases e tabelle, ci si trova a lavorare con immagini, audio o documenti di testo.
\end{itemize}
Con il tempo a queste si sono aggiunte due ulteriori \textbf{V}, più inerenti alla sfera produttiva:

\begin{itemize}
	\item \textbf{Veracity}: ovvero la misura di fiducia che abbiamo nei dati a disposizione. Data la varietà di sorgenti cui si attinge e la velocità con cui le informazioni possono variare nel tempo, non è raro che tali fattori intacchino l'integrità dell'analisi ed è, dunque, fondamentale considerare tale eventualità e tenerne conto, cercando di limitarla quanto più possibile.
	
	\item \textbf{Value}: il valore delle informazioni che tali dati portano con sè. Un ambiente Big Data richiede investimenti e risorse ed è cruciale assicurarsi che lo sforzo valga la pena e che sia possibile far fruttare la ricerca effetuata.
\end{itemize}

\section{Information Explosion}
Negli ultimi tempi più che mai il processo di datificazione di ogni aspetto, sia sociale che produttivo, è stato il fulcro dell'evoluzione tecnologica. Stiamo vivendo una nuova rivoluzione tecnologica, con eventi rilevati da ogni tipo di sensori e dispositivi, che si tratti dello smartphone di cui tutti ormai disponiamo o della rete di sensori applicata alle varie catene di produzione.\\
Particolarmente rilevante, però, è l'importanza che il settore ha guadagnato negli ultimi anni, con un considerevole aumento della quantità di dati strutturati a disposizione. Proprio in questo modo è possibile ottenere il miglior margine dall'enorme mole di informazioni immagazzinata, riducendo al minimo lo spreco di risorse.\\
Le applicazioni sono, dunque, molteplici e già estremamente diffuse:
\begin{itemize}
	\item Ottimizzazione dei processi produttivi
	
	\item Assistenza sanitaria
	
	\item Business e consulting
	
	\item Social Networks
\end{itemize}

\section{Dai Dati all'Analisi}
Definite le caratteristiche e potenziale relative a questa grande quantità di dati bisogna capire cosa sia possibile ottnere da essi. L'analisi dei dati è stata soggetto nel corso degli anni di una continua evoluzione, che ha portato ad aumentarne le potenzialità sempre più.\\
Possiamo ad oggi distinguere quattro livelli di analisi, differenziati sulla base del obbiettivo temporale su cui concentriamo la nostra attenzione:

\begin{itemize}	
	\item \textbf{Analisi descrittiva}: ("Cosa è successo ?") utilizza il dato per descrivere ciò che è successo nel passato. È l'approccio più classico e al momento nell'analisi dati.
	
	\item \textbf{Analisi diagnostica}: ("Perché è successo ?") punta a trovare le cause che hanno condotto al dato attuale.
	
	\item \textbf{Analisi predittiva}: ("Cosa succederà ?") cerca di osservare il dato per prevedere il comportamento futuro del sistema. È importante sottolineare che nessuna analisi potrà dare una stima totalmente sicura e, perciò, ogni risultato viene considerato con il relativo grado di affidabilità. 
	
	\item \textbf{Analisi prescrittiva}: ("Come influenzare ciò che succederà ?") passa al livello successivo, tentando di influenzare gli esiti futuri a partire dal dato. Rappresenta il futuro dell'analisi dati. 
	
\end{itemize}
Tra queste alternative quella che è più legata al concetto di sistema basato su Big Data è l'analisi predittiva, distinguendosi dalla classica Business Intelligence, più concentrata solitamente su un'analisi di tipo descrittivo e diagnostico.